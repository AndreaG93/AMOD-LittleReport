\documentclass[10pt,a4paper]{article}
\usepackage[utf8]{inputenc}
\usepackage{amsmath}
\usepackage{amsfonts}
\usepackage{amssymb}
\usepackage{algorithm}
\usepackage[noend]{algpseudocode}
\usepackage{adjustbox}
\usepackage{listings}
\usepackage{xcolor}
\usepackage{frontespizio} 

\lstset{
language=Java,
basicstyle=\small\ttfamily,			
keywordstyle=\color{blue},
commentstyle=\color{gray},			
stringstyle=\color{black},			
numbers=left,						
numberstyle=\tiny,					
stepnumber=1,						
breaklines=true						
}

\begin{document}

% Frontespizio
% ------------------------------------------------------
\begin{frontespizio} 
\Universita{Roma ``Tor Vergata'' } 
\Logo[3cm]{logo}
\Facolta{Ingegneria} 
\Corso[Laurea Magistrale]{Ingegneria Informatica} 
\Annoaccademico{2018--2019} 
\Titolo{The Cutting-Stock Problem}
\Sottotitolo{Algoritmi e Modelli per l'Ottimizzazione Discreta}
\NCandidato{Studente} 
\Candidato[0273395]{Andrea Graziani} 
\NRelatore{Docente}{} 
\Relatore{Andrea Pacifici} 
\end{frontespizio} 

\tableofcontents
\newpage


\section{The Cutting-stock problem}

The \textbf{Cutting-stock problem} (\textbf{CSP}) is the problem concerning cutting standard-sized pieces of stock material, called \textit{rolls}, into pieces of specified sizes, \textbf{minimizing material wasted}.

In order to formalize that problem as an \textbf{integer linear programming} (\textbf{ILP}), suppose that stock material width is equal to $W$, while $m$ customers want $n_i$ rolls, each of which is wide $w_i$, where $i = 1,...,m$. Obliviously $w_i \leq W$.

\subsection{\textit{Kantorovich} ILP Formulation }



\begin{equation}\label{eqn:P1}
\begin{array} {lllrr} 

(P1) & min & \displaystyle\sum_{k \in K} y_k && \\
& s.t. & \displaystyle\sum_{k \in K} x_i^k \geq n_i, & i = 1,...,m & (demand) \\
&& \displaystyle\sum_{i = 1}^m w_i x_i^k \leq W y_k, & \forall k \in K & (width limitation) \\\\
&& x_i^k \in \mathbb{Z}_{+}, y_k \in \lbrace 0, 1 \rbrace &&
\end{array}
\end{equation}



% ########################################################################################## %
\subsection{\textit{Gilmore} and \textit{Gomory} ILP formulation}

\begin{itemize}
\item An alternative, stronger formulation, is due to Gilmore and Gomory, which main idea is \textbf{to enumerate all possible raw cutting patterns}. 

\item A pattern $j \in J$ is described by the vector $(a_{1j},a_{2j},...,a_{mj})$, where $a_{ij}$ represents the number of final rolls of width $w_i$ obtained from cutting a raw roll according to pattern $j$. 

\item In this model we have an integer variable $x_j$ for each pattern $j \in J$, indicating how many times pattern $j$ is used; in other words it represents how many raw rolls are cut according to pattern $j$. 
\end{itemize}


\begin{equation}\label{eqn:P2}
\begin{array} {lllrr} 

(P2) & min & \displaystyle\sum_{j = 1}^{n} x_j && \\
& s.t. & \displaystyle\sum_{j = 1}^{n} a_{ij} x_j \geq n_i, & i = 1,...,m & (demand) \\\\

& & x_j \in \mathbb{Z}_{+}, j = 1,...,n  &&
\end{array}
\end{equation}


\begin{equation}\label{eqn:LPMP}
\begin{array} {lllrr} 

(LPMP) & min & \displaystyle\sum_{j = 1}^{n} x_j && \\
& s.t. & \displaystyle\sum_{j = 1}^{n} a_{ij} x_j \geq n_i, & i = 1,...,m & (demand) \\\\
& & x_j \in \mathbb{R}_{+}, j = 1,...,n  &&
\end{array}
\end{equation}

Recall that the dual problem of (LPM) is

\begin{equation}\label{eqn:DLPM}
\begin{array} {lllr} 
(DLPMP) & max & \displaystyle\sum_{i = 1}^{m} n_i\pi_i & \\
& s.t. & \displaystyle\sum_{i = 1}^{m} a_{ij}\pi_i \leq 1 & j \in P, \\\\
&& \pi_i \geq 0, i = 1,...,m &
\end{array}
\end{equation}

\ref{eqn:DLPM}


The 
which is called Linear Programming Master Problem. The solution to (LPM) could be fractional, It is possible to round up the fractional solution to get a feasible solution to 

\vspace{5mm}
Where $n$ represents the total number of cutting patterns satisfying following relations:
\begin{equation}
\begin{array} {c} 
\displaystyle\sum_{i=1}^m w_i a_{ij} \leq W \\\\ a_{ij} \in \mathbb{Z}_{+}
\end{array}
\end{equation}


\begin{itemize}
\item \textbf{How many cutting patterns exist?}

\vspace{5mm}
In general, the number of possible patterns \textbf{grows exponentially} as a function of $m$ and it can easily run into the millions. So, it may therefore become impractical to generate and enumerate all possible cutting patterns. 

\vspace{5mm}
Even if we had a way of generating all existing cutting pattern, that is all columns, the \textbf{standard simplex algorithm} will need to calculate the reduced cost for each non-basic variable, which is, from a computational point of view, impossible when $n$ is huge because is very easy for any computer to go \textbf{out of memory}.
\end{itemize}


\begin{itemize}
\item Fortunately, another better approach, which we have used in our project, exists: \textbf{column generation method}.
\end{itemize}

This method solves the cutting-stock problem by starting with just a few patterns and it generates additional patterns when they are needed. To be more precise, the new patterns are found by solving an auxiliary optimization problem, which, as we will see, is a knapsack problem, using dual variable information from the linear problem. 

Auxiliary knapsack problems can be resolved efficiently in $O(mW)$ time using dynamic programming or branch and bound method.



\begin{algorithm}

\caption{}\label{alg:accessControlAlgorithm1}

\begin{algorithmic}

\Function{Column Generation Algorithm }{$\textit{arrivalJob}$}

LPMP

Start with initial columns A of (LPM). For instance, use the simple pattern to cut a roll into bW/wic rolls of width wi, A is a diagonal matrix.



\If {$(n_1 + n_2=N)$}
	\State Send $\textit{arrivalJob}$ on the cloud.
\Else 	
 	\State Send $\textit{arrivalJob}$ on the cloudlet.
\EndIf

\EndFunction

\end{algorithmic}

\end{algorithm}

\clearpage
\newpage
\section{Computational Model}

Let's start now the description of our resolver implementation in which, as known, we will turn all mathematical variables described above into a collection of data structure, classes and variables that, collectively, are able to resolve CSP problem.

Our resolver is been implemented using Java programming languages, which source code is fully available on GitHub\footnote{Source code available on \texttt{https://github.com/AndreaG93/AMOD-Project}}, famous web-based hosting service for version control using \texttt{git}; we remind that \LaTeX\ source code of this report is available too.\footnote{See \texttt{https://github.com/AndreaG93/AMOD-Report}}

In order to properly describe how our implementation works we need to understand how every mathematical elements, seen in previous section, are represented and managed.

\subsection{The CSP instance}

A \textit{generic} CSP instance has been implemented and represented using a Java class called \texttt{CuttingStockInstance}, shown in Listing \ref{code:instance}. In addition to several getter and setter methods, that class has two very important features: 

\begin{itemize}
\item It has got a \texttt{final double} type field, called \texttt{maxItemLength}, which is used to hold the stock material width value that is, in other word, the raw roll width $W$.

\item Through a list data structure, it holds a reference to all items required by customers through an \texttt{ArrayList<>} type field, called \texttt{items}. 

As you can see from Listing \ref{code:instance}, that list contains $m$ \texttt{CuttingStockItem} type instances which, obliviously, represent all specific items request by each customer. To be more precise, every \texttt{CuttingStockItem} type instances $i$, where $i = 1,...,m$, contains two \texttt{double} type fields, called \texttt{length} and \texttt{amount}, which are used respectively to hold a reference to $w_i$ and $n_i$, that is the length and the amount of rolls required by $m$-th customer. 
\end{itemize}

\begin{lstlisting}[frame=lines, caption={\texttt{CuttingStockInstance} class implementation.}, label={code:instance}]
public class CuttingStockInstance {

    private final double maxItemLength;
    private ArrayList<CuttingStockItem> items;

    public CuttingStockInstance(double maxItemLength) {
        this.maxItemLength = maxItemLength;
        this.items = new ArrayList<>();
    }

    public void addItems(double amount, double length) {
        this.items.add(new CuttingStockItem(amount, length));
    }

    double getMaxItemLength() {
        return maxItemLength;
    }

    ArrayList<CuttingStockItem> getItems() {
        return items;
    }
}
\end{lstlisting}

\section{ILP implementation}

In our computational model, a \textit{generic} ILP is been represented using a Java \texttt{interface} called \texttt{LinearProblem}, which code is shown in \ref{code:LinearProblem}. In order to resolve a CSP, our computational model 



Note that 

That interface exports only necessary methods for CSP resolution. Our resolver is unaware of because 





In fact, to ensure low coupling and high cohesion among classes, neither \texttt{SimulationEvent} or \texttt{ComputationalModel} class need to know any specific system information. 



\begin{lstlisting}[frame=lines, caption={\texttt{LinearProblem} interface implementation.}, label={code:LinearProblem}]
public abstract class LinearProblem {

    public abstract LinearProblemSolution getSolution() throws Exception;
    
	public abstract LinearProblemSolution getDualSolution() throws Exception;

    public abstract void setObjectiveFunctionType(LinearProblemType type) throws Exception;

    public abstract void changeObjectiveFunctionCoefficients(double[] newCoefficients) throws Exception;

    public abstract void addConstraint(double[] coefficients, MathematicalSymbol symbol, double value) throws Exception;

    public abstract void setVariables(int totalNumberOfVariables, double lowerBound, double upperBound, VariableType varType) throws Exception;

    public abstract void setObjectiveFunction(double[] coefficients, LinearProblemType type) throws Exception;

    public abstract void addNewColumn(double newVariableLowerBound, double newVariableUpperBound, double value, VariableType varType, double[] columnCoefficient) throws Exception;

    public abstract double[] getColumnCoefficient(int index) throws Exception;
}
\end{lstlisting}

\section{Column Generation Algorithm implementation}

The most important code is included, instead, into \texttt{CuttingStockProblem} class because it contains next-event simulation logic. Technically, \texttt{ComputationalModel} provides \texttt{perform} method implementation, which contains the algorithm\footnote{Lawrence M. Leemis, Stephen K. Park, \textit{Discrete-Event Simulation: A First Course} (Pearson; 1 edition January 6, 2006), Algorithm 5.1.1, page 189} used to perform a simulation based on next-event approach, whose Java implementation is reported in Listing \ref{code:perform}. 


CuttingStockProblem type instance has several very important responsibilities:

\begin{itemize}

\item Allocation and management of all \texttt{SystemComponent} type objects LinearProblem

\item Methods implementation providing for simulation clock and system nodes state variables initialization.

\item Event scheduling management through some methods invocations like \texttt{sche\-dule\-Event\-OnCloud}, \texttt{scheduleEventOnCloulet} etc.

\item Class 2 job interruption management through the so called \texttt{removeCloud\-let\-Class2JobDeparture} method.
\end{itemize}



\begin{lstlisting}[frame=lines, caption={\texttt{solve()} method implementation.}, label={code:solve}]
public void solve() throws Exception {

   buildMasterProblem();
   buildKnapsackSubProblem();

   timer.schedule( task, 15000 );
   long start = System.currentTimeMillis();
        
   executeColumnGenerationAlgorithm();
        
   long finish = System.currentTimeMillis();
   buildSolution();

   this.cuttingStockSolution.setTimeElapsed(finish - start);
}
\end{lstlisting}


\begin{lstlisting}[frame=lines, caption={\texttt{executeColumnGenerationAlgorithm()} method implementation.}, label={code:cga}]
private void executeColumnGenerationAlgorithm() throws Exception {

    LinearProblemSolution masterProblemDualSolution;
    LinearProblemSolution knapsackSubProblemSolution;

    while (!this.timeOut) {

       this.masterProblemSolution = this.masterProblem.getSolution();
       masterProblemDualSolution = this.masterProblem.getDualSolution();

       this.cuttingStockSolution.addObjectiveFunctionValue(this.masterProblemSolution.getValueObjectiveFunction());

       this.knapsackSubProblem.changeObjectiveFunctionCoefficients(masterProblemDualSolution.getSolutions());
       knapsackSubProblemSolution = this.knapsackSubProblem.getSolution();

       if (1 - knapsackSubProblemSolution.getValueObjectiveFunction() < 0) {

           double[] newColumn = knapsackSubProblemSolution.getSolutions();

           this.masterProblem.addNewColumn(0.0, GRB.INFINITY, 1.0, VariableType.REAL, newColumn);
           this.cuttingStockSolution.increaseTotalNumberOfColumnsAdded();

       } else
           break;
       }

    this.timer.cancel();
}
\end{lstlisting}

\begin{lstlisting}[frame=lines, caption={\texttt{buildMasterProblem()} method implementation.}, label={code:cga}]
private void buildMasterProblem() throws Exception {

    ArrayList<CuttingStockItem> cuttingStockItems = instance.getItems();
    double maxItemLength = instance.getMaxItemLength();
    int numberOfVariables = cuttingStockItems.size();

    double[] coefficientObjectiveFunction = new double[numberOfVariables];

    this.masterProblem.setVariables(numberOfVariables, 0, GRB.INFINITY, VariableType.REAL);

    Arrays.fill(coefficientObjectiveFunction, 1);

    this.masterProblem.setObjectiveFunction(coefficientObjectiveFunction, LinearProblemType.min);

    for (int index = 0; index < numberOfVariables; index++) {

        CuttingStockItem currentItem = cuttingStockItems.get(index);

        double[] constraintCoefficients = new double[numberOfVariables];

        constraintCoefficients[index] = ((int) (maxItemLength / currentItem.getLength()));
        this.masterProblem.addConstraint(constraintCoefficients, MathematicalSymbol.GREATER_EQUAL, currentItem.getAmount());
    }
}
\end{lstlisting}

\end{document}