\documentclass[10pt]{beamer}
\usetheme[
%%% option passed to the outer theme
%    progressstyle=fixedCircCnt,   % fixedCircCnt, movingCircCnt (moving is deault)
  ]{Feather}
  
% If you want to change the colors of the various elements in the theme, edit and uncomment the following lines

% Change the bar colors:
%\setbeamercolor{Feather}{fg=red!20,bg=red}

% Change the color of the structural elements:
%\setbeamercolor{structure}{fg=red}

% Change the frame title text color:
%\setbeamercolor{frametitle}{fg=blue}

% Change the normal text color background:
%\setbeamercolor{normal text}{fg=black,bg=gray!10}

%-------------------------------------------------------
% INCLUDE PACKAGES
%-------------------------------------------------------

\usepackage[utf8]{inputenc}
\usepackage[italian]{babel}
\usepackage{helvet}
\usepackage{listings}
\usepackage{xcolor}
\usepackage{graphicx}
\usepackage{amsmath}

%-------------------------------------------------------
% DEFFINING AND REDEFINING COMMANDS
%-------------------------------------------------------

% colored hyperlinks
\newcommand{\chref}[2]{
  \href{#1}{{\usebeamercolor[bg]{Feather}#2}}
}

%-------------------------------------------------------
% INFORMATION IN THE TITLE PAGE
%-------------------------------------------------------

\title[] % [] is optional - is placed on the bottom of the sidebar on every slide
{ % is placed on the title page
      \textbf{Progetto del corso Ingegneria di Internet e del Web a.a. 2017-2018}
}

\subtitle[Progetto del corso Ingegneria di Internet e del Web a.a. 2017-2018]
{
      \textbf{Formulation and Solution Approaches}
}

\author[Andrea Graziani - matricola 0189326]
{      Andrea Graziani - matricola 0273395 \\
      {}
}

\institute[]
{
      Facoltà di Ingegneria Informatica \\
      Università degli Studi di Roma Tor Vergata \\
  
  %there must be an empty line above this line - otherwise some unwanted space is added between the university and the country (I do not know why;( )
}

\date{\today}

% ----------------------------------------------------------------------------------------- %
% Usato per personalizzare l'ambiente 'listings'...
% ----------------------------------------------------------------------------------------- %
\lstset{
language=C,
basicstyle=\tiny\ttfamily,			
keywordstyle=\color{blue},
commentstyle=\color{gray},			
stringstyle=\color{black},			
numbers=left,						
numberstyle=\tiny,					
stepnumber=1,						
breaklines=true						
}

%-------------------------------------------------------
% THE BODY OF THE PRESENTATION
%-------------------------------------------------------

\begin{document}

%-------------------------------------------------------
% THE TITLEPAGE
%-------------------------------------------------------

{\1% % this is the name of the PDF file for the background
\begin{frame}[plain,noframenumbering] % the plain option removes the header from the title page, noframenumbering removes the numbering of this frame only
  \titlepage % call the title page information from above
\end{frame}}

% ----------------------------------------------------------------------------------------- %
\section{Problem definition}
\begin{frame}{The Cutting-stock problem}{Definition}
% ----------------------------------------------------------------------------------------- %

\begin{itemize}
\item The \textbf{Cutting-stock problem} (\textbf{CSP}) is the problem concerning cutting standard-sized pieces of stock material, called \textit{rolls}, into pieces of specified sizes, \textbf{minimizing material wasted}.
\end{itemize}

\begin{itemize}
\item In order to formalize that problem as an \textbf{integer linear programming} (\textbf{ILP}), suppose that stock material width is equal to $W$, while $m$ customers want $n_i$ rolls, each of which is wide $w_i$, where $i = 1,...,m$. Obliviously $w_i \leq W$.
\end{itemize}

\end{frame}

% ----------------------------------------------------------------------------------------- %
\subsection{IP Formulation}
\begin{frame}{The Cutting-stock problem}{ILP Formulation }




\end{frame}

\begin{frame}{The Cutting-stock problem}{\textit{Kantorovich} ILP Formulation }


\begin{tabular}{lllrr}

$(P1)$ & $min$ & $\displaystyle\sum_{k \in K} y_k$ && \\
&$s.t.$ & $\displaystyle\sum_{k \in K} x_i^k \geq n_i,$ & $i = 1,...,m$ & (demand) \\
&& $\displaystyle\sum_{i = 1}^m w_i x_i^k \leq W y_k,$ & $\forall k \in K$ & (width limitation) \\\\
&& $x_i^k \in \mathbb{Z}_{+}, y_k \in \lbrace 0, 1 \rbrace$ &&
\end{tabular}

\end{frame}

% ########################################################################################## %
\begin{frame}{The Cutting-stock problem}{\textit{Gilmore} and \textit{Gomory} ILP formulation}

\begin{itemize}
\item An alternative, stronger formulation, is due to Gilmore and Gomory, which main idea is \textbf{to enumerate all possible raw cutting patterns}. 

\item A pattern $j \in J$ is described by the vector $(a_{1j},a_{2j},...,a_{mj})$, where $a_{ij}$ represents the number of final rolls of width $w_i$ obtained from cutting a raw roll according to pattern $j$. 

\item In this model we have an integer variable $x_j$ for each pattern $j \in J$, indicating how many times pattern $j$ is used; in other words it represents how many raw rolls are cut according to pattern $j$. 
\end{itemize}


\end{frame}

\begin{frame}{The Cutting-stock problem}{\textit{Gilmore} and \textit{Gomory} ILP formulation}
\begin{tabular}{lllrr}

$(P2)$ & $min$ & $\displaystyle\sum_{j = 1}^{n} x_j$ && \\
& $s.t.$ & $\displaystyle\sum_{j = 1}^{n} a_{ij} x_j \geq n_i,$ & $i = 1,...,m$ & (demand) \\\\

& & $x_j \in \mathbb{Z}_{+}, j = 1,...,n $ &&
\end{tabular}

\vspace{5mm}
Where $n$ represents the total number of cutting patterns satisfying following relations:
\begin{equation}
\begin{array} {c} 
\displaystyle\sum_{i=1}^m w_i a_{ij} \leq W \\\\ a_{ij} \in \mathbb{Z}_{+}
\end{array}
\end{equation}

\end{frame}

\begin{frame}{The Cutting-stock problem}{\textit{Gilmore} and \textit{Gomory} ILP formulation}

\begin{itemize}
\item \textbf{How many cutting patterns exist?}

\vspace{5mm}
In general, the number of possible patterns \textbf{grows exponentially} as a function of $m$ and it can easily run into the millions. So, it may therefore become impractical to generate and enumerate all possible cutting patterns. 

\vspace{5mm}
Even if we had a way of generating all existing cutting pattern, that is all columns, the \textbf{standard simplex algorithm} will need to calculate the reduced cost for each non-basic variable, which is, from a computational point of view, impossible when $n$ is huge because is very easy for any computer to go \textbf{out of memory}.
\end{itemize}

\end{frame}

\begin{frame}{The Cutting-stock problem}{\textit{Gilmore} and \textit{Gomory} ILP formulation}

\begin{itemize}
\item Fortunately, another better approach, which we have used in our project, exists: \textbf{column generation method}.
\end{itemize}

This method solves the cutting-stock problem by starting with just a few patterns and it generates additional patterns when they are needed. To be more precise, the new patterns are found by solving an auxiliary optimization problem, which, as we will see, is a knapsack problem, using dual variable information from the linear problem. 

Auxiliary knapsack problems can be resolved efficiently in $O(mW)$ time using dynamic programming or branch and bound method.

\end{frame}



{\1
\begin{frame}[plain,noframenumbering]
  \finalpage{Grazie per l'attenzione!}
\end{frame}}


\end{document}